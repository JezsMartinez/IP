\documentclass[12pt,a4paper]{article}
\usepackage[a4paper, margin=2cm]{geometry}
\usepackage{graphicx}
\graphicspath{{./images/}}
\usepackage{subcaption}
\newcommand{\tr}{\mathrm{tr}}
\usepackage{hyperref}
\usepackage{authblk}
\usepackage[backend=biber, style=chem-acs]{biblatex}

\bibliography{article}
\addbibresource{article.bib}


\begin{document}
\title{Ionizations of Liquid Water from Charged-cell Periodic Subsystem DFT and Embedded Coupled Cluster Simulations}
\author[1]{Jessica Martinez}
\author[1]{Pablo Ramos}
\affil[1]{Department of Chemistry, Rutgers University, Newark, New Jersey}
\author[2]{Andre Gomes}
\affil[2]{Université de Lille, CNRS, UMR 8523 – PhLAM – Physique des Lasers, Atomes et Molécules, Lille, France}
\author[3]{Johannes Tölle}
\affil[3]{Theoretische Organische Chemie, OrganischChemisches Institut and Center for Multiscale Theory and Computation (CMTC), Westfälische Wilhelms-Universität Münster, Corrensstraße 40, 48149, Münster, Germany}
\author[1]{Michele Pavanello\thanks{\textit{Email: m.pavanello@rutgers.edu}}}
\date{}
\setcounter{Maxaffil}{0}
\renewcommand\Affilfont{\itshape\small}
\begin{titlepage}
  \maketitle
\end{titlepage}

\begin{abstract}
Modeling the ionization potential (IP) and electron affinity (EA) of liquid water is challenging for
two reasons: (1) the bulk-like nature of the liquid imposes the use of periodic boundary conditions
(PBCs), which pose roadblocks when considering charge systems; (2) quantitative electronic structure
methods, such as coupled cluster, are generally not available in PBCs. In this work, we tackle both
challenges by employing subsystem DFT to split the extended system into a collection of finite
subsystems embedded by extended, infinite subsystem. This is achieved by an impurity
model ~\cite{tolle2019charged} where high ab initio method as coupled cluster wavefunctions can be introduced to evaluate
the water molecules’ energy functionals. 

The liquid’s electronic structure is expressed in subsystem contributions by invoking nonadditive density 
functionals whereby the total energy of the liquid is expressed as the sum of molecule-additive
and nonadditive contributions ~\cite{krishtal2015subsystem}. The inter-molecular interaction is
split into Coulumb interactions, and such nonadditive terms as the noninteracting
kinetic energy and the noninteracting exchange-correlation ~\cite{krishtal2015subsystem}. 
These contributions represent interactions related among others to exchange, van der Waals and
Pauli repulsion and are all bifunctional of the subsystem densities ~\cite{tolle2019charged}. 

The final IP/EA values reproduce the experimental values to within 0.5 eV and are determined averaging 
over 256 water molecules (or subsystems) considered in the simulation cell, calculated by the energy difference of the
neutral and the polarized system (Called SCF method) ~\cite{bagus1965self,waskom2017mwaskom}. 
\end{abstract}


\section{Introduction}

The high-accuracy calculations of ionization potentials (IPs) and electron affinities (EAs) of condensed-phase molecular systems as liquid water has 
represented a challege for the last years in both experimental and computational fields
\cite{tolle2019charged,gaiduk2018electron,gaiduk2016photoelectron,seidel2016valence}. Thus, ionized states of the electrons take part in many
crucial processes in electrochemistry, photochemistry as well as peculiar states of matter as excess electrons solvated in liquid water
\cite{ambrosio2017electronic}. The most recent theoretical report of IP and EA values \cite{ziaei2018probing} shows that using quasi-particle self-consistent GW calculations and implicit vertex correction in many-body perturbation (MBPT) remarkably affect the EA of liquid water previously reported
\cite{gaiduk2018electron} by Gaiduk et al by about 1eV. Therefore, they through self-consistent GW approach with an implicit vertex correction based on the projector augmented wave (PAW) method, which is the most accurate pseudo-potential \cite{dal2014pseudopotentials} (PPs available), and combined with Bethe–Salpeter equation established values of 10.2 and 1.1 eV, for IP and EA of liquid water, respectively. \\

Following the trend and now including the use of periodic boundary conditions (PBC) another method was introduced \cite{tolle2019charged}
to determine the ionization potentials (IPs) and electron affinities (EAs) of liquid water. Based on determining successfully a quatum-mechanical
model for charged species in PBC which is able to face the complications related to the long-ranged nature of the Coulomb Kernel
$ w(r,r^{'}) = \frac {1}{|r - r^{'}|}$, which decays to zero when two charges are far away in real space. Indeed, it was achieved using an impurity model with two remarkable qualities: 1) The charged periodic system is replaced by a non-periodic one which is still truly
extended (ie., of infinite size) and 2) The potentials of the neutral and the charged system are pegged to a common reference. \\

The former requires an ad hoc mapping of the infinite system (periodic) onto finite number of finite subsystems (non-periodic subsystems) and an
extended (infinite) subsystem, using a formally exact density embedding method, subsystem DFT \cite{wesolowski2015frozen}. Meanwhile, achieving
the latter only requires finding a consistent choice for the $G = 0$ component of the Coulomb Kernel in a reciprocal space. 
The use of the reciprocal space is only chosen for avoiding the complication of dealing with convergent integrals \cite{martin2004electronic}
In addition, in PBC the physical Coulomb kernel is only the one represented in a reciprocal space $w(G) = \frac {4\pi}{|G|^2}$ where
$G \epsilon {\rm I\!R}^3$. Even though, when $G=0$ the Coulomb kernel becomes undefined the total charged density $\rho (G)$ is zero for $G = 0$
, which is not problematic.\\

Here we present an update of the current state-of-the-art of liquid water IPs and EAs base on an impurity model using an exact density
embedding method, subsystem DFT. This may allow us to contribute to the discussion generated about the most accurate value for the EAs of bulk water
by using ab initio electronic structure \cite{ambrosio2017electronic, gaiduk2018electron} by GW approximations. We begin with a brief theoretical
framework description and follow on the description of the EAs and IPs of 256 bulk liquid water.


\section{Theoretical Background}

\subsection{Mapping a periodic system into a collection of non-periodic subsystems and one periodic subsystem}

To cast DFT in a subsystem fashion, we invoke nonadditive functionals in which each energy term of the supersystem is expressed as the sum of
additive and nonadditive contributions ~\cite{martyna1999reciprocal}. Therefore, when dealing with a finite subsystem with electron density
$\eta_I$, and an infinite or extended subsystem with electron density $\eta$ - $\eta_I$, the total density can be defined as the sum of the
finite subsystem and the extended subsystem, and the total energy is given by,

\begin{equation}
	E_{tot} = {E_[{\eta}_I]} + {E_[{\eta} - {\eta}_I]} + {E^{int}[{\eta}_I, {\eta} - {\eta}_I] } 
\end{equation}

The interaction energy can be broken down into the following contributions,

\begin{equation}
	E^{int} = E^{int}_H + V^{int}_eN + T^{nad}_s + E^{nad}_{xc} 
\end{equation}

The two Coulombic terms $E^{int}_H$ and $V^{int}_eN$ are the electron-electron and electro-nuclear interactions, respectively. And the
nonadditive terms $T^{nad}_s$ and $E^{nad}_{xc}$, represent the noninteracting kinetic energy of the system and the noninteracting
exchange-correlation functionals, respectively ~\cite{krishtal2015subsystem}. The two last terms represent interactions related among other to exchange, van der Waals and Pauli repulsion and are all bifunctional of the two subsystem densities.

\subsection{Coulomb interaction energy determination}

As the electron-nuclear interaction can be treated in an equivalent way that the Hartree terms of the interaction energy, with the definition of
the latter, we can map the complete theoretical background. The Hartree interaction energy then is defined between finite and infinite
electronic system by,

\begin{equation}
	E^{int}_H = \int_{{\rm I\!R}^3} dr \int_{\Omega} dr' \frac{1}{|r-r'|} [\eta(r)-\eta_{I}(r)]\eta_{I}(r') 
\end{equation}

The integral over $dr'$ takes place only over a finite volume $\Omega$, here represented by the simulation cell. Contrariwise, the integral in $dr$
is carried out over the entire space, due to the density $\eta -\eta_{I}$ is extended.

To describe the periodic potential interaction with a finite charge, first, we define the finite potential (define by an overbar) as being the potential which uses $\eta_{I}$ (the electron density of the finite subsystem), as,

\begin{equation}
	\bar{v} [\eta_I](r) = \int_{\Omega} dr' \frac{1}{|r-r'|} \eta_{I}(r')
\end{equation}

For equations $(3)$ and $(4)$ the integral over $dr'$ is carried out only over a finite volume $\Omega$ (typically the simualtion cell) because we expect $\eta_{I}(r)=0$ when $r \not\in \Omega$. 

Therefore, computing the potential generated by the periodic system and using that in the computation of the interaction energy $E_{int}$,
which is defined as,

\begin{equation}
	E_{int} = \int_{\Omega} dr [v[\eta](n)-\bar{v}[\eta_{I}](r)]\eta_{I}(r)
\end{equation}

Where $v[\eta](n)$ is the full periodic potential. The potentials are evaluated separately, in which the finite potential is commonly solve in real 
space because $\eta_{I}$ is often the electron density of a finite system. 

To compare total energies from periodic calculations the Coulomb kernel needs to be correctly referenced. In this work particularly, it is achieved 
imposing the $G=0$ value of the periodic Coulomb kernel to match the same limit of the Coulomb kernel of a reference system. Here we choose the
finite system to be the reference system. Thus, we referenced the periodic potential of the extended subsystem ($\eta -\eta_{I}$) to the corresponding
$G=0$ component in the the ionized and neutral systems. 

\subsection{Embedding Scheme for the Neutral System}

For the neutral system, the Coulomb interaction energy can be expressed as a potential that maps the interaction of an accurately infinitely extended
environment onto and isolated subsystem $I$,

\begin{equation}
	v^{I}_{emb} [\eta] (r) = v[\eta](r) - \bar{v} [\eta_I](r)
\end{equation}

Where $v[\eta](r)$ is the total Coulomb potential of the system and $\bar{v} [\eta_I](r)$ the potential of the isolated subsystem $I$. The latter was
evaluated using the Martyna-Tuckerman method whereby density $\eta_I$ is assumed to be isolated and not periodic \cite{martyna1999reciprocal}.
The embedding potential for the neutral subsystem can also be calculated directly from equation $3$.

\subsection{Impurity Model for the Ionized system}

To obtain the embedding potential of a charged subsystem, we consider the system to be composed of an ionized subsystem embedded in a nonionized
environment. To assemble the appropriate embedding potential, first was evaluated a screening potential, 
$v^{screen}[\eta_I](r) = v[\eta_I](r) - \bar{v} [\eta_I](r)$, understood as the electrostic potential of the periodically repeating water
molecules have on a single lattice site, achiving the removal of only one isolated subsystem. The later $\bar{v} [\eta_I](r)$ is evaluated by
Martyna-Tuckerman method\cite{martyna1999reciprocal}. The embedding potential depends on three densities: $\eta$ the total system,
$\eta_I$ the neutral subsystem, and $\eta^{'}_I$ the charged subsystem, and has the form,

\begin{equation}
	v^I_{emb,imp}[\eta, \eta^{'}_I, \eta_I](r) = v[\eta, \eta^{'}_I, \eta_I](r) - \bar{v}[\eta^{'}_I](r)
\end{equation}

with

\begin{equation}
	v[\eta, \eta^{'}_I, \eta_I](r) = v[\eta](r) + \Delta{v}^{screen}[\eta_I, \eta^{'}_I](r)
\end{equation}

where $\Delta{v}^{screen}[\eta_I, \eta^{'}_I](r) = {v}^{screen}[\eta_I](r) - {v}^{screen}[\eta^{'}_I](r)$, which replace the
electrostatic environment given by periodic images of the ionized subsystem having charged density $\rho^{'}_I$ with the neutral charged
density $\rho_I$ of the nonionized subsystem.

\section{Computational Section}

Embedding potentials for ionized and nonionized systems were obtained with embedded Quantum ESPRESSO (eQE) \cite{genova2017eqe}
employing ultrasoft pseudopotentials from PSL pseudopotential library \cite{corso2014comput}. PBE exchange-correlation functional 
\cite{perdew1996phys} was used to evaluate the additive and nonadditive contributions to the total energy, and revAPBEK
\cite{laricchia2011generalized} to the nonadditive Kinetic energy functional. Following previously benchmarking studies
\cite{genova2016avoiding, genova2017cooperation} a total of 40 Ry and 400 Ry were set as the energy cutoffs for the plane wave 
expansions of the molecular orbitals and the charge density, respectively.  

The ground state calculation of each subsystem with the corresponding neutral and polarized embedding potential was determined through ADF
\cite{te2001chemistry} software. We selected each water molecule to be one subsystem (i.e, 256 subsystem in total) according to the most beneficial
error cancelation, previously determined
\cite{genova2016avoiding, kevorkyants2013calculating, pavanello2011modelling, ramos2016critical, solovyeva2012spin},
between the nonadditive kinetic energy and the nonadditive exchange-correlation functionals employed.

A comparison among Density Functional Theory (DFT), using GGA (PBE) \cite{perdew1996phys}, Hybrid (B3LYP) \cite{hertwig1997parameterization},
Double-Hybrid (B2KPLYP \cite{yu2013intermolecular}, B2NCPLYP \cite{yu2014double}, and REVDSDBLYP \cite{kozuch2010dsd}),
and Statistical average of orbital potentials(SOAP) model\cite{schipper2000molecular, gritsenko1999approximation}, 
Møller–Plesset perturbation theory (MP2) \cite{head1988mp2} and HF \cite{marshall1961unrestricted} level of theory was made, with the basis QZ4P.
The final IP/EA values are calculated using an average for each of the 256 water molecules or subsystems considered in the simulation cell,
calculated by the energy difference of the neutral and the polarized system (Called $\Delta$SCF method) \cite{bagus1965self,waskom2017mwaskom}.

\section{Results}
\subsection{IPs of Bulk water}

\begin{figure}[!h]
	\captionsetup[subfigure]{labelformat=empty}
	\centering
	\begin{subfigure}{0.4\linewidth}
		\includegraphics[width=\linewidth]{IP-mp2-hf}
		\caption{MP2 vs HF}
	\end{subfigure}
	\begin{subfigure}{0.4\linewidth}
		\includegraphics[width=\linewidth]{IP-mp2-dft}
		\caption{MP2 vs DFT}
	\end{subfigure}
	\caption{Distribution of IPs of bulk liquid water. The area subtenended by the lines sums up to 64(ie, the number of subsystems). Left: Comparison between MP2 and HF. Right: Comparison between MP2 and DFT}
\end{figure}

\begin{table}[!h]
 \begin{center}
  \caption{average IPs}
  \label{tab:table1}
  \begin{tabular}{ |c|c| }
	\hline\hline
	&IP average (eV)\\
	\hline
	MP2&9.9941\\
	\hline
	HF& 7.8250\\
	\hline
	PBE&9.6089\\
        \hline
        B3LYP&9.5147\\
        \hline
        SAOP&9.5897\\
	\hline
   \end{tabular}
 \end{center}
\end{table}

\section{Conclusions}

\nocite{*}

\printbibliography

\section{Acknowledgments}


\end{document}
